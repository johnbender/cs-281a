\documentclass[usletter]{article}
\usepackage{graphicx}
\usepackage{amsfonts}
\usepackage{amsthm}
\usepackage{amsmath}
\usepackage{amssymb}
\usepackage{test}
\usepackage{complexity}
\usepackage[margin=1.5in]{geometry}
\usepackage{algpseudocode}

\newcommand{\halt}[0]{
  \overline{\textsf{HALT}}
}

\begin{document}

\makeheader{John Bender}{March 20, 2014}{2}{Final Examination}

I have frequently referred to the textbook \cite{textbook} and my notes throughout the course of working on these problems. If necessary I can provide specifics.

\begin{enumerate}
  \item A language $L$ is \textit{Turing-recognizable} if there is a Turing machine that halts iff the input is in $L$. Prove that not all languages are Turing-recognizable.
    \begin{proof}
      Consider $\halt$ and suppose for the sake of contradiction that there is a Turing machine $M_{\halt}$ that halts when a TM \textit{does not} halt for a given input. Now, if $\alpha = \lfloor M_{\halt} \rfloor $ then $M_{\halt}(\alpha, \alpha)$ halts only when $M_{\halt}$ doesn't halt on itself. A contradiction.
    \end{proof}

  \item Give a polynomial-time algorithm for determining whether a given 2-CNF formula is satisfiable.

    \begin{algorithm}
      \begin{algorithmic}[1]
        \Procedure{2SAT}{$\phi$}
        \ForAll{$C_1 \in \phi$}
        \State $x_1 = C_1[1]$
        \State $x_2 = C_1[2]$ \\
        \ForAll{$C_2 \in \phi$}
        \State $y_1 = C_2[1]$
        \State $y_2 = C_2[2]$
        \If{$y_1 = \overline{x_1} \land y_2 = \overline{x_2}$}
        \State \textbf{return} 0
        \EndIf
        \EndFor
        \EndFor \\
        \State \textbf{return} 1
        \EndProcedure
      \end{algorithmic}
    \end{algorithm}

\end{enumerate}

\newpage

\bibliographystyle{abbrv}
\bibliography{1}

\end{document}
