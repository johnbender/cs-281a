\documentclass[usletter]{article}
\usepackage{graphicx}
\usepackage{amsfonts}
\usepackage{amsthm}
\usepackage{amsmath}
\usepackage{amssymb}
\usepackage{test}
\usepackage{complexity}
\usepackage[margin=1.5in]{geometry}
\usepackage{algpseudocode}

\newcommand{\halt}[0]{
  \overline{\textsf{HALT}}
}

\begin{document}

\makeheader{John Bender}{March 20, 2014}{2}{Final Examination}

I have frequently referred to the textbook \cite{textbook} and my notes throughout the course of working on these problems. If necessary I can provide specifics.

\begin{enumerate}
  \item A language $L$ is \textit{Turing-recognizable} if there is a Turing machine that halts iff the input is in $L$. Prove that not all languages are Turing-recognizable.
    \begin{proof}
      Consider $\halt$ and suppose for the sake of contradiction that there is a Turing machine $M_{\halt}$ that halts when a TM \textit{does not} halt for a given input. Now, if $\alpha = \lfloor M_{\halt} \rfloor $ then $M_{\halt}(\alpha, \alpha)$ halts only when $M_{\halt}$ doesn't halt on itself. A contradiction.
    \end{proof}

  \item Give a polynomial-time algorithm for determining whether a given 2CNF formula is satisfiable.

    The basic idea is that for a 2CNF to be satisfiable, given any two clauses, one should not be constructed of the negation of the other's variables.

    Note, seemingly only covers one way in which a 2CNF might by unsatisfiable.

    \begin{algorithm}
      \begin{algorithmic}[1]
        \Procedure{2SAT}{$\phi$}
        \ForAll{$C_1 \in \phi$}
        \State $x_1 = C_1[1]$
        \State $x_2 = C_1[2]$ \\
        \ForAll{$C_2 \in \phi$}
        \State $y_1 = C_2[1]$
        \State $y_2 = C_2[2]$ \\
        \If{$y_1 = \overline{x_1} \land y_2 = \overline{x_2}$}
        \State \textbf{return} 0
        \EndIf
        \EndFor
        \EndFor \\
        \State \textbf{return} 1
        \EndProcedure
      \end{algorithmic}
    \end{algorithm}

  \item Prove that there is a real number $0 < p < 1$ ...

    Idea: We know that it is possible to simulate a $p$ biased coin with a fair coin by checking against the binary expansion of $p$. So, with high probability, by Markov's inequality, it is possible with some reordering to generate enough ones and zeros using $\approx 3n$ flips of $p$ biased coin to approximate the expansion of $p$ to $n$ bits.

    Further, we can make the following assertion about some $p$:
    \begin{equation*}
      \{ y : y\ \text{is the binary expansion of}\ p\text{, s.t.}\ y\ \text{encodes a pair}\ \langle M, x \rangle\ \text{where}\ M\ \text{halts on}\ x \}
    \end{equation*}

    This is similar to how we know that a linear space circuit family exists that decides a similar subset of the halting problem.

  \item Prove that the polynomial hierarchy collapses if TQBF has a polynomial-time randomized algorithm.

    \begin{proof}
      We know the following:

      \begin{enumerate}
        \item $\BPP \subseteq \Sigma_{2}^{p} \cap \Pi_{2}^{p}$
        \item $\PH \subseteq \PSPACE$
        \item TQBF is $\PSPACE$-complete
      \end{enumerate}

      As a result, if TQBF has polynomial time randomized algorithm then by (a) and assumption TQBF is in $\Sigma_{2}^{p} \cap \Pi_{2}^{p}$ and by (b) and (c) all languages in $\PH$ can be reduced to TQBF in polynomial time collapsing $\PH$.
    \end{proof}

  \item Two players take turns choosing vertices from a directed graph ...

    The basic idea is to ``roll-up'' the result of each path explored in a depth first manner.

   \begin{itemize}
     \item[1.] Pick a node at random. Record the fact that this is move 1.
     \item[2.] Explore a random edge and increment the move number each time. Do this all the way until a node has no edges to follow (i.e. depth first).
     \item[3.] If the move number is odd then it's a win for the first player and even is a win for the second player.
     \item[4.] Move ``back'' up the tree to the parent node and record the win or loss for the first player.
     \item[5.] \textbf{Reuse} the space that was used for the last edge traversal to traverse another edge.
     \item[6.] Place each result at the interior nodes in disjunction with previous results. When there are no more edges move back up the tree and repeat.
   \end{itemize}

   When all edges from the root node have been explored, if the disjunction there is true then there is a way for the first player to force a win.

   There are three components to the space being used by the algorithm at any point in time. The information being recorded at each node and the depth of the path taken. Clearly the number of vertices in a path can be at most $n$. Also the record of all edges to traverse at a given node (for ensuring we don't traverse the same path twice) can be at most $n-1$. Finally the size of the boolean formula used in interior nodes of the graph can involve at most $n$ values.

 \item Show that the parity of $n$ bits can be computed by an $\{\land, \lor, \neg \}$-circuit of polynomial size and depth $O(\log n)$.

   We choose a binary tree where the leafs are the variables and the interior nodes are xor gates built using the construction shown in class. Also we balance the tree to a power of two for $n > 1$ using additional variables set to zero.

   \begin{claim}
     If there are an odd number of ones this circuit will produce a 1 otherwise 0.
   \end{claim}

   \begin{proof}
     We proceed by induction on the structure of the circuit.

     Base: Clearly at the leaf nodes if the variable is 1 then the circuit will also output 1 and 0 otherwise.

     Induction: Suppose that the claim holds for sub-trees $x$ and $y$ then we need to consider each possible combination.

     Case $x = 1$ and $y = 1$: xor of the output is 0 and adding two odds makes an even.

     Case $x = 0$ and $y = 0$: xor of the output is 0 and adding two evens makes an even.

     Case $x = 1$ and $y = 0$: xor of the output is 1 and adding an odd and an even is an odd and a similar argument holds for the reverse.
   \end{proof}

   Since this is a perfect binary tree the $height = \log_2 (n')$ where $n'$ is the $n$ leaves ``filled in'' to the nearest power of two.

   Since $n'$ is at most double $n-1$, the number of nodes for a perfect binary tree is $2^{height+1}-1$ \cite{wikipedia}, and each xor ``gate'' has a constant size we know that the circuit has polynomial size in $n$ variables.

   \item Prove that changing the completeness requirement from $2/3$ to $1$ in the definition of $\IP$ results in the same complexity class.

     \begin{proof}
      We know the following:

      \begin{enumerate}
        \item $\IP = \PSPACE$.
        \item TQBF has a protocol that has perfect completeness.
        \item TQBF is $\PSPACE$-complete.
      \end{enumerate}

      By (a) and (c) we know that all problems in $\IP$ can be reduced to TQBF. Then, by (b), all problems in $\IP$ have a polynomial time verifier with a completeness probability of 1 because each verifier can first reduce to TQBF and then proceed with its protocol in polynomial time.
    \end{proof}

    \item Given a system $L$ of linear equations over $\mathbb{F}_2$ ...

      We know the following:

      \begin{enumerate}
        \item solving a system of linear equations, that is $\mathsf{opt}(L) = |L|$, requires polynomial time.
        \item It's possible to reduce a restricted subset of 3SAT formulae that use XOR to a system of linear equations in $\mathbb{F}_2$
      \end{enumerate}

\end{enumerate}


\newpage

\bibliographystyle{abbrv}
\bibliography{1}

\end{document}
